% !TEX root = ../thesis.tex

\chapter{Úvod}\label{ch:introduction}


%Úvod práce stručne opisuje stanovený problém, kontext problému a motiváciu pre riešenie problému. Z úvodu by malo byť jasné, že stanovený problém doposiaľ nie je vyriešený a má zmysel ho riešiť.

%V úvode neuvádzajte štruktúru práce, t.j. o čom je ktorá kapitola. Rozsah úvodu je minimálne 2 celé strany (vrátane formulácie úlohy).

\section*{Formulácia úlohy}

%V rámci úvodu tiež vysvetlite vaše chápanie úlohy, ktorú budete riešiť. Prvú verziu tejto časti by ste mali napísať čo najskôr, aby ste sa uistili, že rozumiete, čo je cieľom vašej práce, a prekonzultovali ju so školiteľom. Neskôr v priebehu práce na projekte ju môžete spresňovať a dopĺňať.

\begin{enumerate}
 
    \item Štúdium teoretických východísk a analýza výsledkov predchádzajúcich prác. Oboznámenie sa s konceptom Multi-access Edge Computing (MEC) a decentralizovaných výpočtových sietí. Detailne preskúma výsledky predchádzajúcich diplomových prác, ktoré sa venovali optimalizácii výpočtov na hrane a bezpečnostným aspektom v 5G a 6G sieťach.
    \item Návrh a realizácia MEC siete v laboratóriu počítačových sietí. Cieľom je vytvoriť funkčné laboratórne prostredie pre MEC sieť, ktorá bude umožňovať efektívne využívanie hraničných výpočtov. Študent navrhne architektúru siete, konfiguráciu zariadení a integráciu výpočtových uzlov na hrane.
    \item Implementácia výpočtov na hrane MEC s využitím robotov Rosmaster R2 (vozidlá autonómnej mobility - robot so štruktúrou Ackermann podporujúci operačný systém ROS, ROS2 pre použitie AI). Návrh a zrealizovanie riešenia umožňujúce robotom Rosmaster R2 využívať výpočtový výkon MEC siete. Riešenie bude optimalizované na zníženie latencie, minimalizáciu spotreby energie a efektívne rozdelenie výpočtovej záťaže medzi hraničné a cloudové uzly.
    \item Testovanie a hodnotenie efektivity navrhnutého riešenia. Súčasťou budú merania výkonnosti siete a analýza parametrov, ako sú latencia, šírka pásma, výpočtový výkon a spotreba energie. Overenie stability systému a jeho schopnosť zvládnuť rôzne typy pracovných záťaží.
    \item Vypracovanie dokumentácie a odporúčaní pre ďalší vývoj. Na záver bude zdokumentovaný celý proces návrhu, implementácie a testovania. Študent vypracuje odporúčania pre ďalší vývoj a optimalizáciu MEC sietí v spojení s autonómnymi robotmi.

\end{enumerate}
