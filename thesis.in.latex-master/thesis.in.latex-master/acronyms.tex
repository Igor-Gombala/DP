% Acronyms
% ========
%
% An acronym is a word formed from the initial letters in a phrase. 
%
% Acronym Definition Exapmle:
% ---------------------------
% \newacronym{gcd}{GCD}{Greatest Common Divisor}
% \newacronym{dry}{DRY}{Don't Repeat Yourself}
%
% Usage:
% ------
% You can use these three options:
% 
% \acrlong{}  
%   Displays the phrase which the acronyms stands for. Put the label of the acronym inside the braces. In the example, \acrlong{gcd} prints Greatest Common Divisor. 
%
% \acrshort{} 
%   Prints the acronym whose label is passed as parameter. For instance, \acrshort{gcd} renders as GCD. 
%
% \acrfull{} 
%   Prints both, the acronym and its definition. In the example the output of \acrfull{dry} is Don't Repeat Yourself (DRY). 
% 
% For more information see:
% -------------------------
% * https://www.sharelatex.com/learn/Glots both, the acronym and its definition.ssaries 
% * https://en.wikibooks.org/wiki/LaTeX/Glossary
%


\newacronym{rcnn}{R-CCN}{Region-Based Convolutional Neural Network}
\newacronym{ssd}{SSD}{Single-Shot Detector}
\newacronym{yolo}{YOLO}{You Only Look Once}
\newacronym{cnn}{CCN}{Convolutional Neural Network}
\newacronym{flops}{FLOPs}{Floating point operations per second}
\newacronym{mec}{MEC}{Multi-access edge computing}
\newacronym{ros}{ROS}{Robot Operating System}
\newacronym{bsd}{BSD}{Berkeley Software Distribution}